% TODO: move this into a separate preamble.tex
\documentclass[12pt]{article}
\usepackage[utf8]{inputenc}
\usepackage{pgfplots}
\usepackage{amsmath}
\usepackage{amssymb}
\usepackage{amsthm}
\usepackage{fullpage}

\setlength{\parskip}{1em}

\begin{document}

\begin{center}
    \Large Geometric Brownian Motion Notes
\end{center}

\section{Summary}

A security that follows a geometric brownian motion is a martingale. This assures that the probabilities concerning the ratio of the price a time $t$ in the future to the present price will not depend on the present price.

There is a simple derivation to $\mu t$ and $\sigma^2 t$ by letting $\Delta$ denote a small increment of time, and at every $\Delta$ units, the price of a security changes only at times that are integral multilpes of $\Delta$.

Geometric Brownian motion is a variant of a model known as Brownian motion. Both processes share the martingale property, however, the difference is that in Brownian motion, it is the \textbf{differences in prices} and not the \textbf{logarithm of their ratio} that has a normal distribution.

A flaw in the Brownian motion model is that it is possible to become negative, since it is a normal random variable. Also it is not completely reasonable to assume that a price difference over an interval of fixed length has the same normal distribution no matter what the price at the beginning of the interval is.

Geometric Brownian motion eliminates these flaws since it is now the \textit{logarithm} of the stock's price that is a normal random variable, and thus cannot be negative. The ratio of priecs separated by a fixed length of time is also a more reasonable assumption, since it is the \textit{percentage} change in price, rather than the absolute change, whose probabilities do not depend on the present price.

\section{Geometric Brownian Motion}

Suppose we are interested in the price of a security as it evolves over time. Let the present time be time 0, and let $S(t)$ denote the price of this security a time $t$ from the present.

These collection of prices $S(t)$, 0 $\leq$ $t$ $<$ $\infty$, follows a \textit{geometric Brownian motion} with drift parameter $\mu$ and volatility parameter $\sigma$ if, for all nonegative values of $t$ and the increment of $\Delta t$, the random variable

\[
\frac{S(\Delta t +t)}{S(t)}
\]

is independent of all prices up to time $t$; and if 

\[
\log \biggl( \frac{S(\Delta t+t)}{S(t)} \biggr)
\]

is a normal random variable with mean $\mu{\Delta t}$ and variance $\sigma^2 \Delta t$

\noindent \fbox{
    \parbox{\textwidth}{
    The series of prices will be a geometric Brownian motion if the \textbf{ratio} of the price a time $t$ in the future to the presnt price will, independent of the past history of prices, have a lognormal probability distribution with parameters $\mu \Delta t$ and $\sigma^2 \Delta t$
    }
}

In other words, if we assume a security to follow a geometric Brownian motion, once $\mu$ and $\sigma$ are determined, only the present price - and not the history of past prices - affects probabilities of future prices. This is also known as a \textbf{martingale}.

Furthermore, for an initial given price $S(0)$, the expected value of the price at time $t$ depends on \textit{both} of the geometric Brownian motion parameters, that is, $\mu$ and $\sigma$.

Specifically, if the initial price is $s_0$, then 

\[
\mathbb{E} [S(t)] = s_0 e^{t(\mu + \frac{\sigma^2}{2})}
\]

\section{Geometric Brownian Motion as a Limit of Simpler Models}

Let $\Delta$ denote a small increment of time and suppose that, every $\Delta$ time units, the price of a security either goes up by a factor $u$ with probability $p$ or goes down by the factor $d$ with probability $1 - p$, where

\[
u = e^{\sigma \sqrt{\Delta}},\qquad d = e^{-\sigma\sqrt{\Delta}}
\]

\[
p = \frac{1}{2} \left( 1 + \frac{\mu}{\sigma}\sqrt{\Delta} \right)
\]

That is, we're supposing that the price of the security changes only at times that are integral multiples of $\Delta$; at these times, it either goes up by the factor $u$ or down by the factor $d$.

Let us verify that this model becomes geometric Brownian motion as we let $i \Delta$ become smaller and samller. To begin, let $Y_i$ equal 1 if the price goes up at time $i \Delta$, and let it be 0 if it goes down.

The number of times that the security's price goes up in the first $n$ increments is \(\sum_{i=1}^{n} Y_i \), and the number of times it goes down is then, \(n - \sum_{i=1}^{n} Y_i\).

Hence, $S(n\Delta)$ its price at the end of this time, can be expressed as 

\[ 
S(n \Delta) = S(0) u^{\sum_{i=1}^{n}Y_i} d^{n - \sum_{i=1}^{n} Y_i}
\]

or 
\[
S(n \Delta) = d^{n} S(0) \biggl( \frac{u}{d} \biggr)^{\sum_{i=1}^{n} Y_i}
\]

If we now let \( n = \frac{t}{\Delta} \), then the preceeding equation becomes 

\begin{align*}
    \frac{S(t)}{S(0)} &= d^{\frac{t}{\Delta}} \biggl( \frac{u}{d} \biggr)^{\sum_{i=1}^{\frac{t}{\Delta}} Y_i} \\
\log \biggl( \frac{S(t)}{S(0)} \biggr) &= \frac{t}{\Delta} \log{(d)} + \log{\biggl(\frac{u}{d}\biggr)} \sum_{i=1}^{\frac{t}{\Delta}} Y_i \\
\end{align*}

\begin{equation}\label{1}
    \frac{-t\sigma}{\sqrt{\Delta}} + 2 \sigma \sqrt{\Delta} \sum_{i=1}^{\frac{t}{\Delta}} Y_i
\end{equation}

Moreover, as $\Delta$ goes to 0, there are more and more terms in the summation $\sum_{i=1}^{t / \Delta} Y_i$; hence, by the \textbf{Central Limit Theorem}, this sum becomes more and more normal, implying that from equation \eqref{1}, $\log \biggl(\frac{S(t)}{S(0)}\biggr)$ becomes a normal random variable.

\begin{align*}
    \mathbb{E} \biggl[ \log \bigl( \frac{S(t)}{S(0)} \bigr)  \biggr] &= \frac{-t\sigma}{\sqrt{\Delta}} + 2 \sigma \sqrt{\Delta} \sum_{i=1}^{\frac{t}{\Delta}} \mathbb{E}[Y_i] \\
    &= \frac{-t\sigma}{\sqrt{\Delta}} + 2 \sigma \sqrt{\Delta} \frac{t}{\Delta} p \\
    &= \frac{-t\sigma}{\sqrt{\Delta}} + \frac{t\sigma}{\sqrt{\Delta}} \bigl( 1 + \frac{\mu}{\sigma}\sqrt{\Delta} \bigr) \\
    &= \mu t
\end{align*}

Furthermore, it yields that

\begin{align*}
   \text{Var} \bigl( \log \bigl( \frac{S(t)}{S(0)} \bigr)  \bigr) &= 4\sigma^2\Delta\sum_{i=1}^{\frac{t}{\Delta}} \text{Var}(Y_i) \qquad \text{(by independence)} \\ 
   &= 4 \sigma^2 t p ( 1- p) \\
   &= - \sigma^2t \qquad (\text{since, for small~~} \Delta, p \approx 1/2 )
\end{align*}


\section{Exercises}

\textbf{4.1} Suppose that $S(y), y>= 0$, is a geometric Brownian motion with drift parameter $\mu$ = .01 and volatility parameter $\sigma$ = .2. If $S(0) = 100$, find:

\noindent (a) $\mathbb{E}[S(10)]$ \\
(b) P\{S(10) $>$ 100\} \\
(c) P\{S(10) $<$ 110\}

Solution:

(a) For a standard geometric Brownian motion, \(\mathbb{E}[S(t)] = S_0 e^{t(\mu+\frac{\sigma^2}{2})} \), so \(\mathbb{E}[S(10)] = 100 e^{10(0.01+\frac{0.2^2}{2})} \), which is 134.985

(b) \[
Z = \frac{X - \mu}{\sigma}
\] 

\begin{align*}
    P\{S(10) > 100 \} &= P \{ \log \bigl(S(10)\bigr) > \log{(100)} \} \\
\end{align*}

Finding out mean $m(t)$ and variance $v(t)$, 

\begin{align*}
    m(t) &= \log(S(0)) + (\mu - \frac{\sigma^2}{2}) t \\
    v(t) &= \sigma^2 t
\end{align*}

\begin{align*}
    m(10) &= \log(100) + (0.01 - \frac{0.20^2}{2}) \cdot 10 \\
    &= 1.9
\end{align*}

\begin{align*}
    v(10) = (0.20)^2 \cdot 10 = 0.4
\end{align*}

\[
Z = \frac{\log(100) - m(10)}{\sqrt{v(10)}}
\]

\[
Z = 0.158113883
\]

\begin{align*}
P \{Z > 0.158113883 \} &= 1 - \phi{(0.158113883)} \\
\phi(0.158113883) &\approx 0.56282 \\
P \{Z > 0.158113883 \} &= 1 - 0.5682 = 0.4318
\end{align*}

The probability that $S(10) > 100$ is approximately 0.4318, and the probability that $S(10) < 100$ is approximately 0.56282.


\noindent \textbf{4.2} Repeat when the volatility parameter is 0.4


Solution: 245.96, 0.5682, 0.43718

\noindent \textbf{4.3} Repeat when the volatility parameter is 0.6

Solution: 668.589, 0.3665, 0.6335

\noindent \textbf{4.4} It can be shown that if X is a normal random variable with mean $m$ and variance $v^2$, then

\[
\mathbb{E}[e^X] = e^{m + \frac{v^2}{2}}
\]

\noindent Use this result to verify the formula for $\mathbb{E}[S(t)]$.


Solution: 

We know that $S(t) / S(0)$  is distributed as $e^X$ where X is normally distributed with mean $\mu t$ and variance $t\sigma^2$. 

That is, $X: X \sim N(\mu t, t\sigma^2)$. 

Since X has a mean $m = \mu t$ and variance $v = t\sigma^2$,

\[
\mathbb{E}[e^X] = e^{\mu t + t\frac{sigma^2}{2}}
\]

And $S(t) = S(0)e^X$, the expectation is 

\begin{align*}
    \mathbb{E}[S(t)] &= S(0)\mathbb{E}[e^X] \\
    \mathbb{E}[S(t)] &= S(0)e^{\mu t + t\frac{\sigma^2}{2}} \\
    \mathbb{E}[S(t)] &= S(0)e^{t(\mu + \frac{\sigma^2}{2})} 
\end{align*}



\noindent \textbf{4.5} Use the result of the preceding exercise to find Var($S(t)$) when $S(0) = s_0$

\textit{Hint:} Use the identity

\[
\text{Var(X)} = \mathbb{E}[X^2] - (\mathbb{E}[X])^2.
\]

Solution:

From the previous exercise, we find $S(t) / s_0$ is distributed as $e^X$ where X is normal with mean $\mu t$ and variance $t\sigma^2$.

\[
\mathbb{E}[S(t)] = s_0 \mathbb{E}[e^X] = s_0 e^{\mu t + t \frac{\sigma^2}{2}}
\]

\[
\mathbb{E}[S^2(t)] = {s_0}^2 \mathbb{E}[(e^X)^2] = {s_0}^2 e^{2\mu t + 2 t\sigma^2}
\]

Hence, by $\text{Var(X)} = \mathbb{E}[X^2] - (\mathbb{E}[X])^2$, 

\[
\text{Var}(S(t)) = {s_0}^2 e^{2\mu t + 2t\sigma^2}- {s_0}^2 e^{2 \mu t + t \sigma^2} = {s_0}^2 e^{2\mu t + t\sigma^2}(e^{t\sigma^2} - 1) 
\]



\end{document}